% Adaptación de la plantilla classicthesis de André Miede
% Compilable con pdflatex

\documentclass[oneside,openright,titlepage,numbers=noenddot,openany,headinclude,footinclude=true,cleardoublepage=empty,abstractoff,BCOR=5mm,paper=a4,fontsize=12pt,ngerman,spanish,american]{scrreprt}
\usepackage[utf8]{inputenc}
\usepackage[T1]{fontenc}
\usepackage{fixltx2e}
\usepackage{graphicx}
\usepackage{grffile}
\usepackage{longtable}
\usepackage{wrapfig}
\usepackage{rotating}
\usepackage[normalem]{ulem}
\usepackage{amsmath}
\usepackage{textcomp}
\usepackage{amssymb}
\usepackage{capt-of}
\usepackage[colorlinks=true]{hyperref}
\usepackage{tikz}
\usepackage{minted}
\input{macros}
\usepackage[T1]{fontenc}
\usepackage[beramono,eulerchapternumbers,linedheaders,parts,a5paper,dottedtoc,manychapters,pdfspacing]{classicthesis}
% General settings
% They will applied only when compiling to pdf.

% Geometry and spacing of paragraphs
\setcounter{secnumdepth}{0}
\usepackage{enumitem}
\setitemize{noitemsep,topsep=0pt,parsep=0pt,partopsep=0pt}
\setlist[enumerate]{topsep=0pt,itemsep=-1ex,partopsep=1ex,parsep=1ex}
\usepackage[top=1in, bottom=1.5in, left=1in, right=1in]{geometry}
\setlength\itemsep{0em}
\setlength{\parindent}{0pt}
\usepackage{parskip}

% Minted package for code listings
\usepackage{minted} \usemintedstyle{colorful}
\setminted{fontsize=\small}
\setminted[haskell]{linenos=false,fontsize=\small}
\renewcommand{\theFancyVerbLine}{\sffamily\textcolor[rgb]{0.5,0.5,1.0}{\oldstylenums{\arabic{FancyVerbLine}}}}
\input{classicthesis-config}
\setcounter{secnumdepth}{3}

% Datos de portada
\author{Nicolas Bourbaki}
\date{\today}
\title{Lorem Ipsum en la filosofía matemática\\\medskip
\large Un enfoque moderno}



\begin{document}

\maketitle
\tableofcontents


\chapter*{Abstract}
Occaecati expedita cumque est. Aut odit vel nobis praesentium dolorem
sed eligendi. Inventore molestiae delectus voluptatibus
consequatur. Et cumque quia recusandae fugiat earum repellat
porro. Earum et tempora vel voluptas. At sed animi qui hic eaque
velit.

Saepe deleniti aut voluptatem libero dolores illum iusto
iusto. Explicabo dolor quia id enim molestiae praesentium sit. Odit
enim doloribus aut assumenda recusandae. Eligendi officia nihil
itaque. Quas fugiat aliquid qui est.

Quis amet sint enim. Voluptatem optio quia voluptatem. Perspiciatis
molestiae ut laboriosam repudiandae nihil.


\ctparttext{
  \color{black}
  \begin{center}
    Esta es una descripción de la parte de matemáticas.
    Nótese que debe escribirse antes del título
  \end{center}
}
\part{Parte de matemáticas}

\chapter{Sección primera}

Lorem ipsum dolor sit amet, consectetur adipiscing elit. Nunc finibus
augue a tellus volutpat, quis cursus mi egestas. Suspendisse cursus
eleifend lacus non consectetur. Proin nibh nisl, tincidunt eu
tincidunt et, auctor vitae mi. Mauris tincidunt finibus enim, sit amet
facilisis mi imperdiet in. Pellentesque pellentesque felis sed
condimentum congue. Nunc at mauris et velit tempor gravida faucibus
sed velit. Sed risus justo, feugiat non accumsan vitae, commodo a
mauris. Vestibulum ac tortor ligula. Donec pulvinar neque ac purus
varius dapibus. Donec fermentum bibendum ultrices. Quisque dictum,
purus quis semper lacinia, justo nibh cursus orci, vitae sagittis
mauris ex aliquet odio. Nullam vel lacus eu tellus dictum tempus in
vel sapien. Morbi pulvinar tincidunt tincidunt. Cras eros tortor,
commodo at tempus a, ultrices vel magna. Nulla eget mauris
arcu. Vivamus aliquet sem odio, non faucibus ante iaculis non.

Aenean aliquet metus nisi, eu congue nulla egestas nec. Etiam pretium
cursus orci a venenatis. Pellentesque placerat feugiat tortor tempus
laoreet. Sed at rutrum leo. Aenean magna nulla, egestas quis efficitur
a, consequat id ligula. Nulla facilisi. Duis orci urna, bibendum vel
mollis non, commodo vitae tellus. Integer luctus ex sed nibh
convallis, id malesuada metus facilisis.

\begin{theorem}
En las condiciones dadas anteriormente se tiene

\[
g(t) = \int^b_a K(t,s)f(s)\ \diff s.
\]
\end{theorem}

Vivamus sit amet sodales elit, in tempus nulla. Nullam euismod rhoncus
quam, ac fringilla massa fermentum vehicula. Maecenas vestibulum purus
non pellentesque congue. Sed nec massa purus. Mauris et quam
dignissim, mollis lacus id, tristique diam. Vestibulum molestie
vehicula tellus quis dignissim. Curabitur accumsan augue ut dictum
semper. Proin tristique quam nulla, vel placerat tortor feugiat
vel. Vestibulum quis ultricies enim, in vestibulum ante.

\chapter{Sección segunda}
Vivamus fringilla egestas nulla ac lobortis. Etiam viverra est risus,
in fermentum nibh euismod quis. Vivamus suscipit arcu sed quam dictum
suscipit. Maecenas pulvinar massa pulvinar fermentum
pellentesque. Morbi eleifend nec velit ut suscipit. Nam vitae
vestibulum dui, vel mollis dolor. Integer quis nibh sapien.

Nullam laoreet augue at erat consectetur fermentum. In interdum
aliquam condimentum. Etiam luctus viverra tellus, a consequat justo
venenatis id. Curabitur a scelerisque erat. Nulla et ante eget lacus
varius accumsan. Suspendisse odio nisl, convallis ac orci nec,
efficitur semper enim. Pellentesque eros ipsum, luctus et viverra
maximus, auctor at est. In hac habitasse platea dictumst. Etiam
ultricies suscipit diam, eget dignissim tellus. Proin justo orci,
volutpat placerat libero a, gravida interdum nisi.

\begin{corollary}
Se obtiene el siguiente resultado

\[\begin{tikzcd}
a\rar{f} \dar[swap]{g} & b \dar{h} \\
c\rar{k} & d
\end{tikzcd}\]
\end{corollary}

\ctparttext{\color{black}\begin{center}
Esta es una descripción de la parte de informática.
\end{center}}

\part{Parte de informática}
\chapter{Sección tercera}
El siguiente código es un ejemplo claro de lo que queremos
ejemplificar.

\begin{minted}[frame=lines]{haskell}
-- From the GHC.Base library.
class  Functor f  where
    fmap        :: (a -> b) -> f a -> f b

    -- | Replace all locations in the input with the same value.
    -- The default definition is @'fmap' . 'const'@, but this may be
    -- overridden with a more efficient version.
    (<$)        :: a -> f b -> f a
    (<$)        =  fmap . const

-- | A variant of '<*>' with the arguments reversed.
(<**>) :: Applicative f => f a -> f (a -> b) -> f b
(<**>) = liftA2 (\a f -> f a)
-- Don't use $ here, see the note at the top of the page

-- | Lift a function to actions.
-- This function may be used as a value for `fmap` in a `Functor` instance.
liftA :: Applicative f => (a -> b) -> f a -> f b
liftA f a = pure f <*> a
-- Caution: since this may be used for `fmap`, we can't use the obvious
-- definition of liftA = fmap.

-- | Lift a ternary function to actions.
liftA3 :: Applicative f => (a -> b -> c -> d) -> f a -> f b -> f c -> f d
liftA3 f a b c = liftA2 f a b <*> c


{-# INLINABLE liftA #-}
{-# SPECIALISE liftA :: (a1->r) -> IO a1 -> IO r #-}
{-# SPECIALISE liftA :: (a1->r) -> Maybe a1 -> Maybe r #-}
{-# INLINABLE liftA3 #-}
{-# SPECIALISE liftA3 :: (a1->a2->a3->r) -> IO a1 -> IO a2 -> IO a3 -> IO r #-}
{-# SPECIALISE liftA3 :: (a1->a2->a3->r) ->
                                Maybe a1 -> Maybe a2 -> Maybe a3 -> Maybe r #-}

-- | The 'join' function is the conventional monad join operator. It
-- is used to remove one level of monadic structure, projecting its
-- bound argument into the outer level.
join              :: (Monad m) => m (m a) -> m a
join x            =  x >>= id
\end{minted}

Vivamus fringilla egestas nulla ac lobortis. Etiam viverra est risus,
in fermentum nibh euismod quis. Vivamus suscipit arcu sed quam dictum
suscipit. Maecenas pulvinar massa pulvinar fermentum
pellentesque. Morbi eleifend nec velit ut suscipit. Nam vitae
vestibulum dui, vel mollis dolor. Integer quis nibh sapien.

\end{document}